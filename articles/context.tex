\chapter{Research Context}

\section{Overview}

the chapter deals with the research context of the thesis, the overarching motivations for each chapter as well as chapter specific motivations.

Context can split into 

- digital musical instrument conservation
- more generally digital conservation

musical instrument conservation deals with the topics relevant to the conservation of musical instrument in particular digital methods and tools to conserve musical instruments digitally

Digital conservatio relates to the detail and problems associated with process of conserving digiital tools

we have an interesting problem in that we are not only conserving some physical object digitally, but the tools used to do so themselves need to be conserved and  sustained in some manner.

For this second point we can look towards the Open Science movement which has been growing since arround 2016, as well as 

\section{Digital object Identifier}

Adoption of diital objectidentifier for referencing software releases. Point towards a particular release of software

Open software is living in the sense that it capable of constant change, when used in research or research sowftare this can be a problem for citeableility and reusablility

\subsection{YIN Example}

By way of example, look at the YIN source code from the paper \cite{cheveigne_yin-a-fundamental_2002} 

source code originally available from IRCAM at \url{http://www.ircam.fr/pcm/cheveign/sw/yin.zip}. 

Later at cited at \url{http://audition.ens.fr/adc/sw/yin.zip} and as recently as 2023 \cite{alain_cancellation_2023} where it is available for, though via unsecured http.

\begin{quote}
The audition team belongs to the Laboratoire des Systèmes Perceptifs. This is the team's informal webpage, with links to some personal pages. See the official page of the LSP for up-to-date information.
\end{quote} from \url{audition.ens.fr}

Official webpage \url{iec-lsp.ens.fr} offline and now \url{lsp.dec.ens.fr}

Just one example of link rot but for an algorithm that is widely cited. DOIs and independent digital repositories would go some way ot address this.


\section{Music Instrument Sustainability}
\subsection{CITES}

Convention on International Trade in Endangered Species of Wild Fauna and Flora (CITES) is a multi-lateral treaty for the protection of animals and plant species that have been put under threat through involvement in trade.

CITES had 3 Appendices that correlate to the level of threat the species is under from Appendix I being the most server to III being the least.

Under the CITES appendices are a number of materials traditionally used within the manufacture of musical instrumemnt. Of note for musical instrument manufacture are the restrictions on:

\begin{itemize}
    \item Dalbergia nigra (Brazilian rosewood)Appendix I  [1992] \url{https://cites.org/sites/default/files/E-CoP17-62-R1.pdf}
    \item Paubrasilia echinata (Pernambuco) Appendix II  [2007] \url{https://cites.org/sites/default/files/documents/E-CoP19-Inf-53_0.pdf}
    \item Madagascar Ebony [2013] Appendix II \url{https://cites.org/sites/default/files/timber_id_materials/files/CITES%20%20Timber%20-%20A%20guide%20to%20CITES-listed%20tree%20species%202023.pdf}
    \item Madagascar Ebony [2013] Appendix II \url{https://cites.org/sites/default/files/timber_id_materials/files/CITES%20%20Timber%20-%20A%20guide%20to%20CITES-listed%20tree%20species%202023.pdf}
    \item Dalbergia spp. (Rosewood) [2017] \url{https://www.aphis.usda.gov/sites/default/files/cites.pdf} All Dalbergia species
\end{itemize}


Pernambuco \cite{toledo_operation_2022} (original portugese \cite{toledo_como_2022}) illegal harvested.

Gibson guitar company 2008 involvement in importing illegally harvested wood in Madagascar in 2008 \cite{justice_gibson_2012}.

Instrument makers involved in these activities shows what impact they have on the industry. 

Many materials used in historic instruments covered under cites. Even if there was no question of the preservation of the material, it is either legally impossible or financially and ecologically uinsustainable to maintain historic instryments in an original form.

Musical minstrument conservation has to contend with a ship of thesues problem. How much can the original instrument be changed before it is considered the same instrument. One benefit of a conservation approach that does not priooritise the playing of instruments is that this problem does to typically need to be addressed.,

\section{Music Instrument Conservation}
\subsection{Digital Museums and Visitor Interaction}
- Context of San colombano / tagkliavini 
- Also benton fletcher, also russel collection
\section{Perception of Musical Instruments}
\subsection{Influence of visual}

Thesis draws upon the research craiied out by Fritz et al between 2012 and 2018 that concerned double blind studies using old cremonese violins, specifically Guarneri del Gesù and Antonio Stradivari.

Through the course of these studies its was shown that player preference fell towards newer violins around 2 years of age.

Study by Tsay \cite{tsay_sight_2013} also explored the impact that sight has over perception. In the course of which they found x and y and z.

Sight interefereing with listening is not a new observation \cite{mcgurk_hearing_1976}, replicated with musicians in 1990 \cite{saidana_mcgurk_1990} and 2019 \cite{politzer_mcgurk_2019}

Trained musicians are not immune to these effcts.

This says soethgi about the natureof musical instrument conservation. If these instruments are degrading both in fragiliyty and sonic qualit, what can be done to preserve them.

Value-based preservation: in this case preservation is the maintaining of instruments in original working order without compromising their constructio. Under that definition musical instruments are fundamentally finite.

This is not the definition of conservation that musical instruments museums will use.

Explore living heritage \cite{poulios_moving_2010} vs and alongside values-based heritage

\section{Open Science}

\cite{unesco_open_2021}

\begin{quote}
    accelerate human progress and foster knowledge societies and highlighting the importance of narrowing the STI and digital gaps existing between and within countries and regions,
\end{quote}

\begin{quote}
Considering that more open, transparent, collaborative and inclusive scientific practices, coupled with more accessible and verifiable scientific knowledge subject to scrutiny and critique, is a more efficient enterprise that improves the quality, reproducibility and impact of science, and thereby the reliability of the evidence needed for robust decision- making and policy and increased trust in science,
\end{quote}


\cite{wilkinson_fair_2016}

\section{What is FAIR}

FAIR \cite{wilkinson_fair_2016} four foundational principles on the

\begin{quote}
Findability, Accessibility, Interoperability, and Reuse of digital assets    
\end{quote}

Summarising \cite{wilkinson_fair_2016}

\begin{itemize}
\item Findability: data are described with rich metadata
\item Accessibility: protocol is open, free, and universally implementable
\item Interoperability: the ability of data or tools from non-cooperating resources to integrate or work together with minimal effort.
\item Reuse: data are released with a clear and accessible data usage license
\end{itemize}

used in academic research. Wilkinson et al. call out scientific data, but the principles should be applicable to all fields of academic research where digital assets are created.

Version control systems such as git provide an easy means of tracking and tagging versions of software and services such as GitHub provide a means to distribute.

perpetual digital repository

A goal to demonstrate the considerations needed to adhere to these four principles, particular within a cross-discpline project such is the topic of this thesis.

All digital assets, be it source code, measurement data or CAD files

\section{Software Sustainability}

The Software sustainability institute at the university of Edinburgh has provided guidance citation \cite{hong_how_2019} of software as well as evaluation of software sustainablity \url{https://www.software.ac.uk/resources/online-sustainability-evaluation}

\subsection{SSI Areas of sustainability}

All projects have strived to adhered to the SSI guidance covering points of:

\begin{itemize}
\item What does your software do
\item Documentation: Instructions on build
\item Plans for the future: outlined in issues
\item Availability of your software: open and available via GitHub
\item Source code structure: structured source code
\item Open standards
\item Building from source: where applicable, build instructions
\item Installing the binary: where applicable, installation instructions or deployment to package manager systems
\item Testing
\item Portability: cross-platform where possible
\item Community
\item Contributor Policy
\item Identity
\item Copyright
\item Licence: GPLv3 Licence
\end{itemize}


\subsection{Addressing Open Science, FAIR, and Software Sustainability}

To address all of the above, the output of this thesis has confirmed to the following. 

All work digital has been put under version control using the git version control system. Version controlled repository are stored on the GitHub platform to allow for public access.

Releases of the software are stored on the Zenodo platform for assigning a DOI

A `release` is defined as any self-contained working version of an asset that was used in the course of research. Digital assets are living and can travel through periods of transition. A release of represents a version that is self-contained, in some final form at a particular moment in time.

All digital assets are available under the GLPv3 licence to provide open permission. Allows for the copy, distribute and modify the source files and that any modifications to that digital source must also be made available under the GPLv3.

\section{Definitions}

\begin{itemize}
    \item CAD: Computer aided design,  Relating to CAD files used in the creation of PCBs or 3D prints
    \item PCB: Printed circuit board. The circuit required for connecting electronics togther. Printed on plate of silicon with a copper coating.
    \item FAIR: Findability, Accessibility, Interoperability, and Reuse of digital assets    
    \item DOI: Digital object Identifier
    \item Digital Asset: Any type of digital data, encapsulating source code, compiled binaries, CAD files, images. Anything created then stored digital and used in the course of research.
    \item FOSS: Free open source software where free encapsulates GNU Project Four Essential Freedoms \cite{gnu_philosophy_2025} the user's ability to run, study, distribute and modify without restriction and that the source code be publicly available.
\end{itemize}


\section{Questions trying to be answered}

\begin{itemize}
    \item how to facilitate the physical experience of playing a musical instrument in a museum collection while preserving the original?
    \item How can the visual of an instrument interface be leveraged to achieve this?
    \item How can digital technologies be used to provide free open open source tools to help help conserve musical instruments?
\end{itemize}
