
%%%%%%%%%%%%%%%%%%%%%%%%%%%%%%%%%%%%%%%%%%%%%%%%%%%%%%%%%%
% Acoustics and Music Technology Final Project Latex Template
%
% CHAPTER 1 PAGE
%
% TOTAL EDITS REQUIRED: 1
%
% NOTE: NO NEED TO INCLUDE ANY FURTHER PREAMBLE IN THIS FILE
%%%%%%%%%%%%%%%%%%%%%%%%%%%%%%%%%%%%%%%%%%%%%%%%%%%%%%%%%%


%%%%%%%%%%%% EDIT %%%%%%%%%%%%
\chapter{Introduction}
\label{chapter1}

The purpose of this paper is to give an introduction to the techniques that are necessary to simulate the sound propagation produced by different radiating bodies in two dimensions.\\
The approach taken to perform these simulations is to produce a model using Finite Difference Time Domain (FDTD) techniques. This kind of technique will allow us to simulate the complete behaviour of the propagation as well as to study its effects and properties.\\
The study will begin by giving a brief introduction to difference operators used in FDTD and the mathematical background necessary to understand the operations that take part in the simulation.
After all the necessary maths is explained, the study will continue by giving an introduction to the 2D wave equation and will develope the equations necessary to simulate the sound propagation using the techniques previously explained.\\
Once these concepts are laid out, an introduction to the different radiating bodies will begin, with particular attention to monopoles, dipoles and quadrupoles. It will be shown how it is possible to simulate the particular behaviour of these radiating bodies by aproximating techniques as well as how they can be introduced into the wave equation.
The study will continue by giving an analysis on accuracy and efficiency of the simulation and it will conclude by gathering all these information and studying its results.\\
The appendix of this text contains the MATLAB code of the studied model.
%%%%%%%%%%%% EDIT %%%%%%%%%%%%
