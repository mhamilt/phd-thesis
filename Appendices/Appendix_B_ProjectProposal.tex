\chapter{Special Project Proposal}
\label{appB}

Physical modelling of room acoustics, as we propose here, intends to provide an accurate description of the propagation behaviour of sound in different environments. These models are developed to help create improved acoustical environments, for a more realistic sound synthesis (Bilbao, 2009). They also prove useful in a multitude of various applications within the field of audio engineering.\\
The specific aim of this project is to develop a two dimensional physical model of the radiation produced by different bodies, using high level programming software such as MATLAB. The radiating bodies in consideration will be represented by monopoles, dipoles, quadrupoles and so on. The reason behind the selection of these bodies is that they can represent different sound sources, such as musical instruments and loudspeakers (Russel, et al., 1999). The sound sources in consideration will be placed in a confined environment, like a small room filled with different obstacles. The behaviour of the radiation is affected by these surroundings will then be analysed.\\ 
The interest of developing this model is to produce the basis from which further models can be produced such as 3D models for room acoustics, loudspeaker enclosures or instrumental bodies.\\
In order to create a physically realistic model it is necessary first to fully understand the theoretical background. Thus, to understand the Mathematics and Physics behind this model, the investigation will begin with a study of the two dimensional wave equation. This study phase will result in the application of a mathematical equation to the desired environment, therefore becoming the backbone from which the model will be developed. 
Once this physical and realistic model of the sound wave propagation is produced, the investigation will continue by further explaining the radiation produced by different sound sources. We will begin with the study of monopoles and dipoles (Russel, et al., 1999) using techniques like Finite Difference Time Domain (FDTD) (Sheaffer, et al., 2012). \\
After introducing these later concepts into the model, we can consider the relevant boundary conditions.
In order to produce an accurate modelling within such conditions, the author will study different techniques such as Finite Element Methods and Boundary Element Methods (Lam, 1999). At first these boundary conditions will reflect an ideal behaviour, allowing the development of a more realistic approach, introducing losses and different absorbing bodies.\\
The finalized model will be able to demonstrate the complete dispersion of sound in the 2D plane, in which different conditions can be implemented. The model will be able to provide a quantifiable explanation of the quality of the environment, as well as determine the conditions for an ideal dispersion. 
I would estimate that this research and development would take around 14 weeks till submission. An approximate timeline outlining the objectives and goals of this project is presented below.\\  
1-2 Weeks:
\begin{itemize} 
	\item Objective: Lay the mathematical foundations, applying them to the desired environment.
	\item Tasks: Research the mathematical and physical background as well as other necessary research (coding, presentation, etc.).
\end{itemize}
2-6 Weeks: 
\begin{itemize}
	\item Objectives: MATLAB code for 2D wave propagation with different sources.
	\item Tasks: 2D wave propagation MATLAB model and implementation of different radiating bodies.
\end{itemize}
6-9 weeks:
\begin{itemize}
	\item Objectives: Full 2D model has been tested and working efficiently.
	\item Tasks: Optimization of MATLAB code as well as implementation of boundary conditions and other refinements.
\end{itemize}
9-12 weeks:
\begin{itemize}
	\item Further develop the application and optimization. Finalise academic report and other necessary paperwork.
\end{itemize}
This project will be supervised by different bodies within the Acoustics and Audio group at the University of Edinburgh. In particular, the main supervisors of this project will be Dr. Stefan Bilbao and PhD student Brian Hamilton.\\
The resources available at any computer lab within the University campus would be sufficient for the performance of this project. Thus, no further reference is necessary in this regard. 
