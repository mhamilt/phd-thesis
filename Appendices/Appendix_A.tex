\chapter{MATLAB Code}
\label{appA}

\begin{lstlisting}
%%%%%%%%%%%%%%%%%%%%%%%%%%%%%%%%%%%%%%%%%%%%
%% 2D Wave Propagation. Radiating Bodies %%%
% Author: Alessandro Rodriguez
% Date: 19th April 2017
%%%%%%%%%%%%%%%%%%%%%%%%%
%%
%%%%%% CHOICE OF SOURCE AND STENCIL %%%%%%%%
source=2;       %1=Monopole ,2=Dipole, 3=Quadrupole
stencil=2;      %1=5point 2=9point
if (stencil~=1 && stencil~=2)
    error('Please select a valid stencil');
elseif (source~=1 && source~=2 && source~=3)
    error('Please select a valid source');
end
%%
%%%%%%%%%% VARIABLES %%%%%%%%%%%%%%%%%%%%%%%
Fs=16000;       %Sample rate
c=343;          %Wave speed
T=1;            %Time of simulation
k=1/Fs;         %Time step
L=5;           %Length of domain
% Derived parameters
if stencil==1
    lambda=1/sqrt(2);       %Stability condition 5 point stencil
else
    alpha=0.74;
    lambda=min(1,(1/(sqrt(2*alpha))));  %Stability 9 point stencil
end
h=c*k/lambda;   %Space Step
N=floor(L/h);   %Number of grid divisions
h=L/N;
lambda=c*k/h;
x=[0:h:L-h];
y=x';
x0=x(floor(N/2));   %X coordinate Source
y0=y(floor(N/2));   %Y coordinate Source
%%
%%%%%%%% DIRAC DELTA APPROXIMATION %%%%%%%%
%%%%%% Threshold 
threshold=10^(-10);
% X coordinate
xden=x-x0;
xIndex=find(abs(xden)<threshold);
dx=sinc((x-x0));
dIx=cos(pi*(x-x0))./((x-x0));
px=1./(x-x0);
dIx(xIndex)=1;
px(xIndex)=1;
% Y coordinates
yden=y-y0;
yIndex=find(abs(yden)<threshold);
dy=sinc((y-y0));
dIy=cos(pi*(y-y0))./((y-y0));
py=1./(y-y0);
py(yIndex)=1;
dIy(yIndex)=1;
%%
%%%%%%%%%%% SOURCE %%%%%%%%%%%%%%%%%%%%%%%%%%
sFreq=220;      %Driving signal f.Frequency (Hz)
if source==1;
    %%%%%%%%%%%% MONOPOLE %%%%%%%%%%%%%%%%%%%%%%%
    source=dy*dx;
elseif source==2;
    %%%%%%%% DIPOLE %%%%%%%%%%%%%%%%%%%%%%%%%%%%%
    theta=pi/2;         %Angle of rotation.
    source=(1/(h^2))*(cos(theta)*(dy*(dIx-(dx.*(px))))...
        +(sin(theta)*(dIy-(dy.*py))*dx));
elseif source==3;
    %%%%%%% QUADRUPOLE %%%%%%%%%%%%%%%%%%%%%%%%%
    theta=pi/4;         %Angle of rotation.
    Qstrengh=0;         %Longitudinal strength
    Qxstrengh=1;        %Lateral strength
    Qxx=Qstrengh*(cos(theta)^2)
	+Qxstrengh*(cos(theta)*sin(theta));
    Qxy=Qstrengh*(cos(theta)*sin(theta))...
        +Qxstrengh*(0.5*(cos(theta)^2-sin(theta)^2));
    Qyx=Qstrengh*(cos(theta)*sin(theta))...
        +Qxstrengh*(0.5*(cos(theta)^2-sin(theta)^2));
    Qyy=Qstrengh*(sin(theta)^2)
	-Qxstrengh*(cos(theta)*sin(theta));

    source=(1/(h^2))*((Qxx*(dy*(-((pi^2)*dx)
	-2*(dIx.*(px))+(2*dx.*(px.^2))))
	+Qxy*((dIy-(dy.*py))*(dIx-(dx.*(px)))))...
        +Qyx*((dIy-(dy.*py))*(dIx-(dx.*(px))))...
        +Qyy*((-((pi^2)*dy)-(2*(dIy.*py)) 
	+(2*dy.*(py.^2)))*dx));
end
%Setting storing values and initial time steps
u=zeros(N,N);
up1=u;
um1=u;
t=0;
while (t<T);
    um1=u;
    u=up1;
    t=t+k;
    %%%%%%%%%%%%%%%%%%%%%%%%%%%%%%%%%%%%%%%%%%%%%%%%%%
    %%% 2D WAVE EQUATION CALCULATION USING FDTD %%%
    if stencil==1 
        %Five point stencil
    up1(2:end-1,2:end-1)=2.*u(2:end-1,2:end-1)
    -um1(2:end-1,2:end-1)
    +(lambda^2.*(u(2:end-1,3:end)...
    +u(2:end-1,1:end-2)+u(1:end-2,2:end-1)...
    +u(3:end,2:end-1)-4.*u(2:end-1,2:end-1)))...
    +source(2:end-1,2:end-1)*sin(2*pi*sFreq*t);

    elseif stencil==2
        %Nine point stencil
    up1(2:end-1,2:end-1)=2.*u(2:end-1,2:end-1)
     -um1(2:end-1,2:end-1)...
    +(lambda^2.*((u(2:end-1,3:end)+u(2:end-1,1:end-2)
    +u(1:end-2,2:end-1)+u(3:end,2:end-1)-4.*u(2:end-1,2:end-1))...
    +0.5*(1-alpha).*(u(3:end,3:end)+u(1:end-2,3:end)
    +u(1:end-2,1:end-2)+u(3:end,1:end-2)...
   -2*(u(2:end-1,3:end)+u(2:end-1,1:end-2)...
    +u(1:end-2,2:end-1)+u(3:end,2:end-1))...
    +4.*u(2:end-1,2:end-1))))...
    +source(2:end-1,2:end-1)*sin(2*pi*sFreq*t);
    end
                 %% PLOT AND MESH %%%
    fig(1)=figure(1);
    surf(u); 
    shading interp
    axis square
    xlabel('x');
    ylabel('y');
    view(2)
    drawnow limitrate;
end
\end{lstlisting}

