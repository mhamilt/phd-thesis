%%%%%%%%%%%%%%%%%%%%%%%%%%%%%%%%%%%%%%%%%%%%%%%%%%%%%%%%%%
% Acoustics and Music Technology Final Project Latex Template
%
% CHAPTER 2 PAGE
%
% TOTAL EDITS REQUIRED: 1
%
% NOTE: NO NEED TO INCLUDE ANY FURTHER PREAMBLE IN THIS FILE
%%%%%%%%%%%%%%%%%%%%%%%%%%%%%%%%%%%%%%%%%%%%%%%%%%%%%%%%%%



\chapter{Measurement: Open Laser Vibrometry}\label{measurement-open-laser-vibrometry}

\section{The Problem}\label{the-problem-1}

\subsection{Vibrometry Theory}\label{vibrometry-theory}

\subsection{Measuring Plates}\label{measuring-plates}

\section{The Solution}\label{the-solution}

\subsection{Designing a vibrometer}\label{designing-a-vibrometer}

\subsection{Design Decisions}\label{design-decisions}

\subsection{Limitations}\label{limitations}

\section{Application}\label{application}

%%%%%%%%%%%% EDIT %%%%%%%%%%%%
\chapter{Finite Difference Time Domain}
\label{chapter2}
%%%REVISE THIS SECTION
The finite-difference time-domain method is a technique which is considered to be one of the most intuitive and simplest ways to approximate a continuous variable function \textit{u(t)} (Bilbao/Schneider) by time series. To approximate this continuous function we say that $u(t_{n})=u(nk)$ for integer $n$, where \textit{k} is considered to be the time step between each value in the time series.\\
In regard to musical applications, these technique yields very interesting and accurate results. This is due to the fact that, as for most audio applications the step time is relatively small. Considering that in audio, the sampling rate is standarized to normally about 44100 Hz, we can see how this time step $k=1/Fs$, will be small enough so that information is preserved with a good degree of accuracy. 

\section{Difference operators}
\label{chapter2:sec1}
In order to produce an approximation to a continuous function through a time series, it is necessary to define a series of operators that are applied to the the approximation. In this case, the operators that are necessary to simulate the function at a particular time will involve different time steps of the approximating series. Therefore for a time series ${u}^{n}$ where $n$ represents the $n^{th}$ time step, it is necessary to introduce the following 
\begin{equation*}
	e_{t+}u^{n} = u^{n+1}  \ \ \ \ \ \  e_{t-}u^{n} = u^{n-1}    \ \ \ \ \       $ e_{t+}e_{t-} = 1
\end{equation*}
These operators are defined as forward and backwards shifts respectively. Thanks to these operators, it is possible to approximate a wide variety of continuous operators, although for the purposes of this paper they will be limited to approximate first and second derivatives.
To produce an approximation to the first derivative operator, it is possible to use different combinations of the backward and forward shifts as follows,
\begin{equation}
	\begin{aligned}
	\delta_{t+} \overset{\Delta}{=} \frac{1}{k} (e_{t+} -1) \cong \frac{d}{dt}
	\end{aligned}
\end{equation}
\begin{equation}
	\begin{aligned}
	\delta_{t-} \overset{\Delta}{=} \frac{1}{k} (1 - e_{t-}) \cong \frac{d}{dt}	
	\end{aligned}
\end{equation}
\begin{equation}
	\begin{aligned}
	\delta_{t\cdot} \overset{\Delta}{=}  \frac{1}{2k} (e_{t+} -e_{t-}) \cong \frac{d}{dt}
	\end{aligned}
\end{equation}
These operators are defined as forward, backward, and centered difference operator respectively.\\ 
It is possible to extend this techinique to higher order derivatives, like for example second order derivatives. To do so, we will require the product of two of the previous difference operators. Hence, the second order difference operator \textit{$\delta_{tt}$} is defined by
\begin{equation}
\label{eqn:ddiff}
	\begin{aligned}
	\delta_{tt} \overset{\Delta}{=} \delta_{t+}\delta_{t-}=\frac{1}{k^2}(e_{t+}-2+e_{t-})\cong\frac{d^2}{dt^2}
	\end{aligned}
\end{equation}  
The reader at this point might have already realized the possibility of combining difference operators to for example approximate second order partial derivatives, this approximation consists on the product of two difference operators acting on different variables. As an example, we will consider the approximation to the double partial derivative $\frac{\partial^{2}}{\partial x \partial y}$,
\begin{equation}
	\begin{aligned}
	\delta_{x+}\delta_{y-}  \overset{\Delta}{=} \frac{1}{k^2} (e_{x+} -1) (1-e_{y-} )=\\
		=\frac{1}{k^2}(e_{x+}-e_{x+}e_{y-}-1+e_{y-}) \cong  \frac{\partial^2}{\partial x \partial y}
	\end{aligned}
\end{equation}


%%%%%%%%%%%% EDIT %%%%%%%%%%%%